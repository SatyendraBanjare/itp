%%%%%%%%%%%%%%%%%%%%%%%%%%%%%%%%%%%%%%%%%%%%%%%%%%%%%%%%%%%%%%%%%%%%%%
% LaTeX Example: Project Report
%
% Source: http://www.howtotex.com
%
% Feel free to distribute this example, but please keep the referral
% to howtotex.com
% Date: March 2011 
% 
%%%%%%%%%%%%%%%%%%%%%%%%%%%%%%%%%%%%%%%%%%%%%%%%%%%%%%%%%%%%%%%%%%%%%%
% How to use writeLaTeX: 
%
% You edit the source code here on the left, and the preview on the
% right shows you the result within a few seconds.
%
% Bookmark this page and share the URL with your co-authors. They can
% edit at the same time!
%
% You can upload figures, bibliographies, custom classes and
% styles using the files menu.
%
% If you're new to LaTeX, the wikibook is a great place to start:
% http://en.wikibooks.org/wiki/LaTeX
%
%%%%%%%%%%%%%%%%%%%%%%%%%%%%%%%%%%%%%%%%%%%%%%%%%%%%%%%%%%%%%%%%%%%%%%
% Edit the title below to update the display in My Documents
%\title{Project Report}
%
%%% Preamble
\documentclass[paper=a4, fontsize=11pt]{scrartcl}
\usepackage[T1]{fontenc}
\usepackage{fourier}

\usepackage[english]{babel}															% English language/hyphenation
\usepackage[protrusion=true,expansion=true]{microtype}	
\usepackage{amsmath,amsfonts,amsthm} % Math packages
\usepackage[pdftex]{graphicx}	
\usepackage{url}


%%% Custom sectioning
\usepackage{sectsty}
\allsectionsfont{\centering \normalfont\scshape}


%%% Custom headers/footers (fancyhdr package)
\usepackage{fancyhdr}
\pagestyle{fancyplain}
\fancyhead{}											% No page header
\fancyfoot[L]{}											% Empty 
\fancyfoot[C]{}											% Empty
\fancyfoot[R]{\thepage}									% Pagenumbering
\renewcommand{\headrulewidth}{0pt}			% Remove header underlines
\renewcommand{\footrulewidth}{0pt}				% Remove footer underlines
\setlength{\headheight}{13.6pt}


%%% Equation and float numbering
\numberwithin{equation}{section}		% Equationnumbering: section.eq#
\numberwithin{figure}{section}			% Figurenumbering: section.fig#
\numberwithin{table}{section}				% Tablenumbering: section.tab#


%%% Maketitle metadata
\newcommand{\horrule}[1]{\rule{\linewidth}{#1}} 	% Horizontal rule

\title{
		%\vspace{-1in} 	
		\usefont{OT1}{bch}{b}{n}
		%\normalfont \normalsize \textsc{} \\ [25pt]
		\horrule{0.5pt} \\[0.4cm]
		\normalsize Interactive Theorem Proving - A. Chlipala (Notes) \\
		\horrule{2pt} \\[0.5cm]
}
\author{
		\normalfont 								\normalsize
        Satyendra Kumar Banjare\\[-3pt]		\normalsize
        \today
}
\date{}

\begin{document}
\maketitle
\section{Lecture 1}
This is a fair introduction to what is the basic idea of interactive theorem proving. We are given some pre conditions 	, post conditions and the core idea/theorem to check. Examples include :
\begin{itemize}
	\item{
		Solving a  simple linear equation for x, eg: $ y = m*x +b $ . The pre conditions are that $m \neq 0 $. Post conditions involve the value of x obtained actually satisfies the original equation. 
	}
	\item{Alias analysis involves determination of optimum strategy to find the number of ways a particular memory address can be accessed. using ITP techniques we can assertain this by checking and case elimination of redundant pointers. 
	}

	\item{Anderson's Analysis}
Often called Anderson-Style Pointer analysis involves the flow sensitive pointer analysis and pointer mutation. It follows assigning a set-notation to pointers bounded by given constraints. It treats all allocations done by one instruction as if they are being done to only 1 object. We define $PT(x)$ as a set that approximats all the locations that can be ppointed by the variable x. Different constraints are generated according to the type of modifications done. We then case-by-case analize them. for better examples : \url{https://www.seas.harvard.edu/courses/cs252/2011sp/slides/Lec06-PointerAnalysis.pdf}. 



\end{itemize}

\end{document}

